\documentclass[12pt,a4paper]{article}
\usepackage{cmap} % Makes the PDF copiable. See http://tex.stackexchange.com/a/64198/25761
\usepackage[T1]{fontenc}
\usepackage[brazil]{babel}
\usepackage[utf8]{inputenc}
\usepackage{amsmath}
\usepackage{amsfonts}
\usepackage{amssymb}
\usepackage{amsthm}
\usepackage[usenames,svgnames,dvipsnames]{xcolor}
\usepackage{hyperref}
\usepackage{multicol}
\usepackage{graphicx}
\usepackage[margin=2cm]{geometry}
\usepackage{cancel}

\hypersetup{
    colorlinks = true,
    allcolors = {blue}
}

% TODO: Consider using exsheets
% http://linorg.usp.br/CTAN/macros/latex/contrib/exsheets/exsheets_en.pdf
%
% http://ctan.org/tex-archive/macros/latex/contrib/exercise/
% Options: answerdelayed,lastexercise,noanswer
\usepackage[answerdelayed,lastexercise]{exercise}

\addto\captionsbrazil{%
\def\listexercisename{Lista de exerc\'icios}%
\def\ExerciseName{Exerc\'icio}%
\def\AnswerName{Solu\c{c}\~ao do exerc\'icio}%
\def\ExerciseListName{Ex.}%
\def\AnswerListName{Solu\c{c}\~ao}%
\def\ExePartName{Parte}%
\def\ArticleOf{de\ }%
}

\renewcommand{\ExerciseHeaderTitle}{(\ExerciseTitle)\ }
\renewcommand{\ExerciseListHeader}{%\ExerciseHeaderDifficulty%
\textbf{%\ExerciseListName\
\ExerciseHeaderNB.\ %
%\ --- \
\ExerciseHeaderTitle}%
%\ExerciseHeaderOrigin
\ignorespaces}
\renewcommand{\AnswerListHeader}{\textbf{\ExerciseHeaderNB.\ (\AnswerListName)\ }}

\newtheorem*{note}{Observação}
\newcommand{\fixme}{{\color{red}(...)}}
\newcommand*\diff{\mathop{}\!\mathrm{d}}
\newcommand*\sen{\operatorname{sen}}
\newcommand*\tg{\operatorname{tg}}
\newcommand*\cotg{\operatorname{cotg}}
\newcommand*\cosec{\operatorname{cossec}}
\newcommand*\cotgh{\operatorname{cotgh}}
\newcommand*\arcsen{\operatorname{arcsen}}
\newcommand*\arctg{\operatorname{arctg}}
\newcommand*\abs[1]{\left|#1\right|}
\newcommand*\R{\mathbb{R}}
\newcommand*\op[1]{\overset{#1}{\rightarrow}}

\renewcommand{\theenumi}{\alph{enumi}}
\renewcommand\labelenumi{(\theenumi) }

\newcommand*\tipo{Prova II}
\newcommand*\turma{PRO112-02U}
\newcommand*\disciplina{CDI2001}
\newcommand*\eu{Helder G. G. de Lima}
\newcommand*\data{23/10/2024}

\author{\eu}
\title{\tipo - \disciplina}
\date{\data}

\begin{document}
\thispagestyle{empty}
\newgeometry{margin=2cm,bottom=0.5cm}
\begin{center}
\includegraphics[width=9.0cm]{marca} \\
\textbf{\tipo\ (\disciplina / \turma)} \\
Prof. \eu\footnote{
Este é um material de acesso livre distribuído sob os termos da licença \href{https://creativecommons.org/licenses/by-sa/4.0/deed.pt_BR}{Creative Commons BY-SA 4.0}}
\end{center}

\noindent Nome do(a) aluno(a): \underline{\hspace{9,7cm}} Data: \underline{\data}

\begin{center}\fbox{
\begin{minipage}{14cm}

{\footnotesize
\begin{itemize}
\renewcommand{\theenumi}{\Roman{enumi}}
\item Identifique-se em todas as folhas.
\item Não é permitido o uso de calculadora.
\item Mantenha o celular e os demais equipamentos eletrônicos desligados durante a prova.
\item Justifique cada resposta com cálculos ou argumentos baseados na teoria estudada.
\item Resolva $4$ das $5$ questões (deixe claro que questão não deverá ser corrigida).
\end{itemize}
}

\end{minipage}
}
\end{center}

\begin{ExerciseList}
\Exercise[title={2,5}] Determine o raio de convergência e o intervalo de convergência de \(\sum_{n=0}^{\infty} \frac{x^n}{(n+1)^n}\).

\Answer Considerando \( a_n = \frac{x^n}{(n+1)^n} \), temos duas opções:

\textbf{Teste da raiz}:
\begin{align*}
    L
    &
    = \lim_{n \to \infty} \sqrt[n]{\left| a_n \right|}
    = \lim_{n \to \infty} \sqrt[n]{\left| \frac{x^n}{(n+1)^n} \right|}
    = \lim_{n \to \infty} \sqrt[n]{\frac{|x^n|}{|(n+1)^n|}}
    = \lim_{n \to \infty} \sqrt[n]{\frac{|x|^n}{(n+1)^n}} \\
    &
    = \lim_{n \to \infty} \frac{|x|}{n+1}
    = |x| \cdot \lim_{n \to \infty} \frac{1}{n+1}
    = |x| \cdot 0
    = 0
    < 1.
\end{align*}

\textbf{Teste da razão}:
\begin{align*}
    L
    &
    = \lim_{n \to \infty} \left| \frac{a_{n+1}}{a_n} \right|
    = \lim_{n \to \infty} \left| \frac{\frac{x^{n+1}}{(n+2)^{n+1}}}{\frac{x^n}{(n+1)^n}} \right|
    = \lim_{n \to \infty} \left| \frac{x^{n+1}}{(n+2)^{n+1}}\right| \left|\frac{(n+1)^n}{x^n} \right|
    = \lim_{n \to \infty} |x| \frac{(n+1)^n}{(n+2)^{n+1}}\\
    &
    = |x| \lim_{n \to \infty} \left( \frac{n+1}{n+2}\right)^n \cdot\frac{1}{n+2}
    = |x| \lim_{n \to \infty} \left(1 - \frac{1}{n+2}\right)^n \cdot\frac{1}{n+2}
    = |x| \lim_{t \to \infty} \left(1 + \frac{1}{t}\right)^{-t-2} \cdot \frac{1}{-t} \\
    &
    = |x| \lim_{t \to \infty}
    \frac{1}{\cancelto{e}{\left(1 + \frac{1}{t}\right)^t}}
    \cdot \frac{1}{\cancelto{1}{\left(1 + \frac{1}{t}\right)^2}}
    \cdot \cancelto{0}{\frac{1}{-t}}
    = 0
    < 1.
\end{align*}

Como ambos os testes fornecem $L = 0 < 1$, a série converge para qualquer \( x \in \mathbb{R} \). Assim, o raio de convergência é \(\boxed{r = \infty}\) e o intervalo de convergência é \(\boxed{I = (-\infty, \infty)}\).

\Exercise[title={2,5}] Verifique se a série \(\sum_{n=1}^\infty \frac{n^{23}}{10n^{2024} + 10}\) é convergente ou divergente.

\Answer Como a série tem termos positivos, aplicamos o teste da comparação. Note que:
\[
\frac{n^{23}}{10n^{2024} + 10}
\leq \frac{n^{23}}{10n^{2024}}
= \frac{1}{10} \cdot \frac{1}{n^{2001}}, \quad \forall n \in \mathbb{N}^*.
\]
Além disso, \(\sum_{n=1}^{\infty} \frac{1}{10} \cdot \frac{1}{n^{2001}}\) converge, pois \(\sum_{n=1}^{\infty} \frac{1}{n^{2001}}\) é uma série hiper-harmônica em que \(p = 2001 > 1\).
Portanto, concluímos que a série original \(\boxed{\sum_{n=1}^{\infty} \frac{n^{23}}{10n^{2024} + 10} \text{ é convergente}}\).


\textbf{Observações}:
\begin{itemize}
\item É preciso usar algum teste de convergência pois a série satisfaz a condição necessária para convergência:
\[
\lim_{n\to\infty} \frac{n^{23}}{10n^{2024} + 10}
= \lim_{n\to\infty} \frac{23n^{22}}{20240n^{2023}}
= \frac{23}{20240} \lim_{n\to\infty} \frac{1}{n^{2001}}
= 0.
\]
\item O teste da comparação não nos permitiria tirar qualquer conclusão ao comparar \(\frac{n^{23}}{10n^{2024} + 10}\) diretamente com \(\frac{1}{n^{2024}}\) já que, embora \(\sum_{n=1}^{\infty} \frac{1}{n^{2024}}\) seja convergente, temos $\frac{n^{23}}{10n^{2024} + 10} > \frac{1}{n^{2024}}$, para $n\geq 2$ (verifique!).
\item O teste da razão não permite decidir sobre a convergência da série, pois é inconclusivo:
\begin{align*}
L
& = \lim_{n \to \infty} \left|\frac{\frac{(n+1)^{23}}{10(n+1)^{2024} + 10}}{\frac{n^{23}}{10n^{2024} + 10}}\right|
= \lim_{n \to \infty} \frac{(n+1)^{23}}{10(n+1)^{2024} + 10} \cdot\frac{10n^{2024} + 10}{n^{23}} \\
& = \lim_{n \to \infty} \left(\frac{n+1}{n}\right)^{23} \cdot \frac{10(n+1)^{2024} + 10}{10n^{2024} + 10}
  = \lim_{n \to \infty} \left(1+\frac{1}{n}\right)^{23} \cdot \frac{10\left(\frac{n+1}{n}\right)^{2024} + \frac{10}{n^{2024}}}{10 + \frac{10}{n^{2024}}} \\
& = \lim_{n \to \infty} \cancelto{1}{\left(1+\frac{1}{n}\right)^{23}} \cdot \frac{\cancelto{10}{10\left(\frac{n+1}{n}\right)^{2024}} + \cancelto{0}{\frac{10}{n^{2024}}}}{10 + \cancelto{0}{\frac{10}{n^{2024}}}}
= 1.
\end{align*}
\end{itemize}

\Exercise[title={2,5}] Mostre que a série \(\sum_{n=1}^{\infty} (-1)^{n-1}\left(\frac{n+1}{n^2}\right)\) é condicionalmente convergente.

\Answer Como a série \(\sum_{n=1}^{\infty} (-1)^{n-1} \left(\frac{n+1}{n^2}\right)\) é alternada, aplicamos o teste de Leibniz:
\begin{itemize}
    \item O termo \(a_n = \frac{n+1}{n^2}\) é positivo e decrescente para \(n \geq 1\), pois
    \begin{align*}
        \frac{n+2}{(n+1)^2} \leq \frac{n+1}{n^2}
        & \Leftrightarrow
        (n + 2)n^2 \leq (n+1)(n+1)^2 \\
        & \Leftrightarrow
        n^3 + 2n^2 \leq n^3 + 3 n^2 + 3 n + 1 \\
        & \Leftrightarrow
        0 \leq n^2 + 3 n + 1, \text{ que é verdadeiro para todo } n\geq 1.
    \end{align*}
    \item \(\lim_{n \to \infty} a_n = \lim_{n \to \infty} \frac{n+1}{n^2}
    = \lim_{n \to \infty} \left(\frac{1}{n} + \frac{1}{n^2}\right) = 0\).
\end{itemize}
Portanto, a série \(\boxed{\sum_{n=1}^{\infty} (-1)^{n-1}\left(\frac{n+1}{n^2}\right) \text{ é convergente}}\). Quanto à convergência absoluta, observe que:
\[
\left|(-1)^{n-1}\left(\frac{n+1}{n^2}\right)\right|
= \left|\frac{n+1}{n^2}\right|
= \left|\frac{1}{n} + \frac{1}{n^2}\right|
= \frac{1}{n} + \frac{1}{n^2}
\geq \frac{1}{n}, \text{ para todo } n \geq 1.
\]
Pelo teste da comparação, como a série harmônica \(\sum_{n=1}^{\infty} \frac{1}{n}\) diverge, concluímos que
\[
\boxed{\sum_{n=1}^{\infty} \left|(-1)^{n-1}\left(\frac{n+1}{n^2}\right)\right| \text{ é divergente}}.
\]
Assim, a série é condicionalmente convergente.


\textbf{Observação}: Se tentássemos analisar a convergência pelo critério da razão, o resultado seria inconclusivo, pois \(\lim_{n\to\infty} \left|\frac{(-1)^{n+1} \left(\frac{n+2}{(n+1)^2}\right)}{(-1)^{n-1} \left(\frac{n+1}{n^2}\right)} \right| = 1\) (verifique!)

\Exercise[title={2,5}] Para cada item abaixo, indique se é verdadeiro (V) ou falso (F).

\emph{Observação 1:} Não é necessário justificar.

\emph{Observação 2:} Cada item marcado errado anula a pontuação de um item correto. Você pode deixar um item em branco para não perder nem ganhar pontos.

\begin{enumerate}
    \item {[0,9 ponto] \bf (\hspace{.7em})} A convergência de \(\sum_{n=1}^{\infty} \frac{\tan(n)}{n^2}\) não pode ser analisada pelo teste da integral.
    \item {[0,8 ponto] \bf (\hspace{.7em})} Se uma série converge absolutamente, então ela converge condicionalmente.
    \item {[0,8 ponto] \bf (\hspace{.7em})} Se \(\{a_n\}_{n=1}^{\infty}\) é uma sequência de termos positivos, então a sequência de suas somas parciais, \(s_n = \sum_{i=1}^n a_i\), é monótona.
\end{enumerate}

\Answer
\begin{enumerate}
    \item \textbf{VERDADEIRO}: A função \(f(n) = \frac{\tan(n)}{n^2}\) não é positiva, nem contínua e nem decrescente em \([1, \infty)\).
    \item \textbf{FALSO}: Para que uma série \(\sum_{n=1}^{\infty} a_n\) seja condicionalmente convergente, \(\sum_{n=1}^{\infty} |a_n|\) precisa ser divergente, enquanto para convergência absoluta, \(\sum_{n=1}^{\infty} |a_n|\) deve convergir.
    \item \textbf{VERDADEIRO}: Como \(s_{n+1} = s_n + a_{n+1}\) e \(a_{n+1} > 0\), \(s_n\) é crescente e monótona.
\end{enumerate}


\Exercise[title={2,5}] Desenvolva a função \(f(x) = xe^x\) em série de Maclaurin e determine seu intervalo de convergência.

\Answer A série de Maclaurin tem a forma \( f(x) = \sum_{n=0}^{\infty} \frac{f^{(n)}(0)}{n!} x^n \). Calculando as derivadas de \(f(x) = xe^x\) em \(x = 0\):
\begin{align*}
f(x) & = xe^x \Rightarrow f(0) = 0,\\
f'(x) & = e^x + xe^x = (x + 1)e^x \Rightarrow f'(0) = 1,\\
f''(x) & = e^x + (x + 1)e^x = (x + 2)e^x \Rightarrow f''(0) = 2,\\
f''(x) & = e^x + (x + 2)e^x = (x + 3)e^x \Rightarrow f'''(0) = 3,\\
\vdots \\
f^{(n)}(x) & = (x + n)e^x \Rightarrow f^{(n)}(0) = n e^0 = n.
\end{align*}
Assim, a série de Maclaurin para \(f(x)\) é:
\[
f(x)
= 0 + \frac{1}{1!} x + \frac{2}{2!} x^2 + \frac{3}{3!} x^3 + \cdots
= \sum_{n=0}^{\infty} \frac{n}{n!} x^n
= \sum_{n=1}^{\infty} \frac{n}{n!} x^n
= \sum_{n=1}^{\infty} \frac{x^n}{(n-1)!}.
\]
Pelo teste da razão,
\begin{align*}
\lim_{n \to \infty} \left| \frac{a_{n+1}}{a_n} \right|
&
= \lim_{n \to \infty} \left| \frac{x^{n+1}}{n!} \cdot \frac{(n-1)!}{x^n} \right|
= \lim_{n \to \infty} \left| \frac{x}{n} \right|
= \lim_{n \to \infty} \frac{|x|}{n}
= |x|\lim_{n \to \infty} \frac{1}{n}
= |x| \cdot 0
= 0 < 1.
\end{align*}
A série converge para qualquer \(x\), com intervalo de convergência \( \boxed{I = (-\infty, \infty)}\).
\end{ExerciseList}

\vfill
\begin{center}
BOA PROVA!
\end{center}

\newpage
\restoregeometry
\section*{Respostas}
\shipoutAnswer
\end{document}
