\documentclass[12pt,a4paper]{article}
\usepackage{cmap} % Makes the PDF copiable. See http://tex.stackexchange.com/a/64198/25761
\usepackage[T1]{fontenc}
\usepackage[brazil]{babel}
\usepackage[utf8]{inputenc}
\usepackage{amsmath}
\usepackage{amsfonts}
\usepackage{amssymb}
\usepackage{amsthm}
\usepackage[usenames,svgnames,dvipsnames]{xcolor}
\usepackage{hyperref}
\usepackage{graphicx}
\usepackage[margin=2cm]{geometry}
\usepackage{cancel}

\hypersetup{
    colorlinks = true,
    allcolors = {blue}
}

% TODO: Consider using exsheets
% http://linorg.usp.br/CTAN/macros/latex/contrib/exsheets/exsheets_en.pdf
%
% http://ctan.org/tex-archive/macros/latex/contrib/exercise/
% Options: answerdelayed,lastexercise,noanswer
\usepackage[answerdelayed,lastexercise]{exercise}

\addto\captionsbrazil{%
\def\listexercisename{Lista de exerc\'icios}%
\def\ExerciseName{Exerc\'icio}%
\def\AnswerName{Solu\c{c}\~ao do exerc\'icio}%
\def\ExerciseListName{Ex.}%
\def\AnswerListName{Solu\c{c}\~ao}%
\def\ExePartName{Parte}%
\def\ArticleOf{de\ }%
}

\renewcommand{\ExerciseHeaderTitle}{(\ExerciseTitle)\ }
\renewcommand{\ExerciseListHeader}{%\ExerciseHeaderDifficulty%
\textbf{%\ExerciseListName\
\ExerciseHeaderNB.\ %
%\ --- \
\ExerciseHeaderTitle}%
%\ExerciseHeaderOrigin
\ignorespaces}
\renewcommand{\AnswerListHeader}{\textbf{\ExerciseHeaderNB.\ (\AnswerListName)\ }}

\newcommand*\diff{\mathop{}\!\mathrm{d}}
\newcommand*\sen{\operatorname{sen}}

\newcommand*\R{\mathbb{R}}

\renewcommand{\theenumi}{\alph{enumi}}
\renewcommand\labelenumi{(\theenumi) }

\newcommand*\tipo{Prova IV}
\newcommand*\turma{NEXM241-A}
\newcommand*\disciplina{CDI2001}
\newcommand*\eu{Helder G. G. de Lima}
\newcommand*\data{03/07/2024}

\author{\eu}
\title{\tipo - \disciplina}
\date{\data}

\begin{document}
\thispagestyle{empty}
\newgeometry{margin=2cm,bottom=0.5cm}
\begin{center}
\includegraphics[width=9.0cm]{marca} \\
\textbf{\tipo\ (\disciplina / \turma)} \\
Prof. \eu\footnote{
Este é um material de acesso livre distribuído sob os termos da licença \href{https://creativecommons.org/licenses/by-sa/4.0/deed.pt_BR}{Creative Commons BY-SA 4.0}}
\end{center}

\noindent Nome do(a) aluno(a): \underline{\hspace{9,7cm}} Data: \underline{\data}

%\section*{Instruções}
\begin{center}\fbox{
\begin{minipage}{14cm}

{\footnotesize
\begin{itemize}
\renewcommand{\theenumi}{\Roman{enumi}}
\item Identifique-se em todas as folhas.
\item Mantenha o celular e os demais equipamentos eletrônicos desligados durante a prova.
\item Justifique cada resposta com cálculos ou argumentos baseados na teoria estudada.
\item Resolva $5$ das $6$ primeiras questões (deixe claro que questão não deverá ser corrigida).
\end{itemize}
}

\end{minipage}
}
\end{center}

%\section*{Questões}
\begin{ExerciseList}
\Exercise[title={2,0}] Encontre o termo geral da sequência de somas parciais da série $\displaystyle \sum_{n = 1}^{\infty} \frac{4}{(n + 1)} \cdot \frac{3}{(n + 2)}$ e determine se a série converge ou diverge, obtendo o valor de sua soma, se possível.
\Answer Considerando que, $a_n = \frac{4}{(n + 1)} \cdot \frac{3}{(n + 2)} = \frac{12}{(n + 1) (n + 2)} = \frac{12}{n + 1} - \frac{12}{n+2}$, a $n$-ésima soma parcial é
\begin{align*}
    s_n
    & = \left(\frac{12}{2} - \cancel{\frac{12}{3}}\right) + \left(\cancel{\frac{12}{3}} - \cancel{\frac{12}{4}}\right) + \ldots + \left(\cancel{\frac{12}{n + 1}} - \frac{12}{n+2}\right)
        = \frac{12}{2} - \frac{12}{n+2}
        = 6 - \frac{12}{n+2}.
\end{align*}
Portanto,
\[
\sum_{n = 1}^{\infty} \frac{12}{(n + 1)(n + 2)} = \lim_{n \to \infty} 6 - \frac{12}{n+2} = 6.
\]

\textbf{Observação}: Uma verificação rápida que poderia ser feita é testar se o termo $a_n$ tende a zero, já que isso é necessário para convergência da série. No caso, obteríamos $\lim_{n \to \infty} \frac{4}{(n + 1)} \cdot \frac{3}{(n + 2)} = 0$, o que torna possível (mas não garante) a convergência da série.

Além disso, também é possível a comparação
\[
     \frac{4}{(n + 1)} \cdot \frac{3}{(n + 2)}
\leq \frac{4}{n} \cdot \frac{3}{n}
= 12 \frac{1}{n^2},
\]
que mostra que a série é convergente (mas não nos fornece a soma), já que $\sum \frac{1}{n^2}$ é convergente.

Por outro lado, os testes da razão e da raiz seriam inconclusivos.


\Exercise[title={2,0}] Encontre o termo geral da sequência de somas parciais da série $\displaystyle \sum_{n = 1}^{\infty} \frac{2^{n-1}}{5^{n + 1}}$ e determine se a série converge ou diverge, obtendo o valor de sua soma, se possível.
\Answer Considerando que, $a_n = \frac{2^{n-1}}{5^{n + 1}} = \frac{1}{5^{2}}\frac{2^{n-1}}{5^{n-1}} = \frac{1}{5^{2}}\left(\frac{2}{5}\right)^{n-1}$, a $n$-ésima soma parcial é
\begin{align*}
    s_n
    & = \frac{1}{5^{2}} \frac{2^{0}}{5^{0}} + \frac{1}{5^{2}} \frac{2^{1}}{5^{1}} + \ldots + \frac{1}{5^{2}}\left(\frac{2}{5}\right)^{n-1}
      = \frac{1}{5^{2}} \left[ \left(\frac{2}{5}\right)^{0} + \left(\frac{2}{5}\right)^{1} + \ldots + \left(\frac{2}{5}\right)^{n-1}\right] \\
\end{align*}
Consequentemente,
\begin{align*}
    \frac{2}{5} s_n
    & = \frac{2}{5} \frac{1}{5^{2}} \left[ \left(\frac{2}{5}\right)^{0} + \left(\frac{2}{5}\right)^{1} + \ldots + \left(\frac{2}{5}\right)^{n-1}\right] \\
    & = \frac{1}{5^{2}} \left[ \left(\frac{2}{5}\right)^{1} + \left(\frac{2}{5}\right)^{2} + \ldots + \left(\frac{2}{5}\right)^{n}\right]
\end{align*}
e
\begin{align*}
    s_n - \frac{2}{5} s_n
      = \frac{3}{5} s_n
    & = \frac{1}{5^{2}} \left[ \left(\frac{2}{5}\right)^{0} + \left(\frac{2}{5}\right)^{1} + \ldots + \left(\frac{2}{5}\right)^{n-1}\right]
      - \frac{1}{5^{2}} \left[ \left(\frac{2}{5}\right)^{1} + \left(\frac{2}{5}\right)^{2} + \ldots + \left(\frac{2}{5}\right)^{n}\right] \\
    & =  \frac{1}{5^{2}} \left[\left(\frac{2}{5}\right)^{0} - \left(\frac{2}{5}\right)^{n}\right]
\end{align*}
Assim, o termo geral da sequência de somas parciais é
\[
s_n = \frac{5}{3} \cdot \frac{1}{5^{2}} \left[1 - \left(\frac{2}{5}\right)^{n}\right]
    = \frac{1}{15} \left[1 - \left(\frac{2}{5}\right)^{n}\right]
\]
Portanto, $\displaystyle \sum_{n = 1}^{\infty} \frac{2^{n-1}}{5^{n + 1}} = \lim_{n \to \infty} \frac{1}{15} \left[1 - \left(\frac{2}{5}\right)^{n}\right] = \frac{1}{15}$.

\textbf{Observação}: Se fosse aplicado o teste da razão ou da raiz, os resultados seriam $\lim_{n \to \infty} \frac{a_{n+1}}{a_n} = \frac{2}{5} < 1$ e $\lim_{n \to \infty} \sqrt[n]{a_n} = \frac{2}{5} < 1$. Em ambos os casos, poderíamos concluir que a série é convergente (mas ainda não saberíamos sua soma).

A soma também pode ser obtida notando que $\sum_{n = 1}^{\infty} \frac{2^{n-1}}{5^{n + 1}} = \sum_{k = 0}^{\infty} \frac{2^k}{5^{k + 2}} = \sum_{k = 0}^{\infty} \frac{1}{25}\left(\frac{2}{5}\right)^k$ é uma série geométrica com termo inicial $\frac{1}{25}$, razão $q = \frac{2}{5}$, e portanto sua soma é $S = \frac{1}{25} \frac{1}{1-\frac{2}{5}} = \frac{1}{15}$.

\Exercise[title={2,0}] Mostre que a série de funções $\displaystyle\sum_{n=1}^\infty \frac{\sen(n x)}{(n+7)^2}$ é convergente, qualquer que seja $x \in \R$.
\Answer Como $\frac{|\sen(n x)|}{(n+7)^2} \leq \frac{1}{n^2+14n+49}\leq \frac{1}{n^2}$, e $\sum_{n=1}^\infty \frac{1}{n^2}$ é uma série convergente (por ser hiper-harmônica com $p=2>1$), segue do critério da comparação que $\displaystyle\sum_{n=1}^\infty \left|\frac{\sen(n x)}{(n+7)^2}\right|$ é convergente, qualquer que seja $x \in \R$, e portanto $\displaystyle\sum_{n=1}^\infty \frac{\sen(n x)}{(n+7)^2}$ é absolutamente convergente. Como convergência absoluta implica convergência, $\displaystyle\sum_{n=1}^\infty \frac{\sen(n x)}{(n+7)^2}$ também é convergente.


\Exercise[title={2,0}] \color{black} Verifique que o teste da integral pode ser aplicado e utilize-o para determinar se a série  $\displaystyle\sum_{n=1}^\infty \frac{4}{2n+3}$ é absolutamente convergente, condicionalmente convergente, ou divergente.
\Answer Seja $f(x) = \frac{4}{2x + 3}$. Então \(f(x)\) é contínua e positiva em \([1, \infty)\). Além disso, \(f^\prime(x) = -\frac{8}{(2x + 3)^2} < 0\) se \(x \in [1, \infty)\), logo \(f(x)\) é decrescente em \([1, \infty)\). Com isso, pode-se aplicar o teste da integral:
\begin{align*}
    \int_1^\infty \frac{4}{2x+3} \diff{x}
    & = \lim_{b \to +\infty} 2\int_1^b \frac{1}{2x+3} \cdot 2 \diff{x}
      = \lim_{b \to +\infty} 2 \int_5^{2b + 3} \frac{1}{u} \diff{u}
      = \lim_{b \to +\infty} 2 \ln|u|\bigg|_{u=5}^{2b+3} \\
    & = \lim_{b \to +\infty} 2 (\ln|2b+3| - \ln 5) = +\infty
\end{align*}

Como a integral imprópria diverge, o critério da integral garante que a série $\displaystyle\sum_{n=1}^\infty \frac{4}{2n+3}$, e portanto $\displaystyle\sum_{n=1}^\infty \left| \frac{4}{2n+3}\right|$, também diverge.


\Exercise[title={2,0}] Determine o intervalo e o raio de convergência da série $\displaystyle\sum_{n=1}^\infty 3^n(nx)^n$.
\Answer Se $a_n = 3^n(nx)^n = 3^nn^nx^n = (3nx)^n$, então $|a_n| = 3^nn^n|x|^n$ e $\sqrt[n]{|a_n|} = 3n|x|$. Logo:
\begin{itemize}
    \item Se $x=0$, $\displaystyle \lim_{n \to \infty} \sqrt[n]{|a_n|} = \lim_{n\to \infty} 0 = 0$.
    \item Se $x>0$, $\displaystyle \lim_{n \to \infty} \sqrt[n]{|a_n|} = \lim_{n\to \infty} 3nx = +\infty$.
    \item Se $x<0$, $\displaystyle \lim_{n \to \infty} \sqrt[n]{|a_n|} = \lim_{n\to \infty} -3nx = -\infty$.
\end{itemize}
Portanto, o intervalo de convergência é $I = \{0\}$ e o raio de convergência é zero.

Alternativamente, se for usado o critério da razão, tem-se:
\[
\frac{|a_{n+1}|}{|a_n|}
= \frac{3^{n+1}(n + 1)^{n+1}|x|^{n+1}}{3^nn^n|x|^n}
= \frac{3^{n+1}}{3^n}
  \frac{(n + 1)^n}{n^n}(n + 1)
  \frac{|x|^{n+1}}{|x|^n}
= 3 \left(1 + \frac{1}{n}\right)^n(n + 1) |x|.
\]
Como $\lim_{n \to \infty} \left(1 + \frac{1}{n}\right)^n = e$, chega-se à mesma conclusão de antes.

\Exercise[title={2,0}] Para cada item abaixo, diga se é verdadeiro (V) ou falso (F).

\emph{Observação 1:} Não é necessário justificar.

\emph{Observação 2:} Cada item marcado errado anula a pontuação de um item marcado corretamente. Você tem a opção de deixar o item em branco e, nesse caso, você nem ganha e nem perde pontos.

\begin{enumerate}
    \item $\boldsymbol{\left(\quad\right)}$ A sequência $\{a_n\}_{n = 1}^\infty$ definida por $a_n = n^2 - 3n + 2$ é monótona não-decrescente.
    \item $\boldsymbol{\left(\quad\right)}$ A série $\sum_{n=1}^\infty \frac{n}{n + 2024}$ converge pois $\frac{n}{n + 2024} < 1$ para todo $n \in \mathbb{N}^*$.
    \item $\boldsymbol{\left(\quad\right)}$ Se a série $\sum_{n=1}^\infty a_n$ é convergente então a sequência $\{a_n\}_{n = 1}^\infty$ é decrescente.
    \item $\boldsymbol{\left(\quad\right)}$ $\frac{\diff}{\diff{x}}\left( \sum_{n=0}^\infty x^{2n}\right) = 2\sum_{n=1}^\infty nx^{2n-1}$.
\end{enumerate}
\rule{\textwidth}{2px}

\Answer
\begin{enumerate}
    \item \textbf{Verdadeiro}. Se $f(x) = x^2 - 3x + 2$ então $f^\prime(x) = 2x - 3 > 0$ para $x > \frac{3}{2}$, logo $f(x)$ é crescente em $(\frac{3}{2}, +\infty)$. Em particular, $a_{n} = f(n) < f(n+1) = a_{n + 1}$ para $n \in \mathbb{N}$, $n \geq 2$. No entanto, $f(1) = 0 = f(2)$, então a sequência $a_n = f(n)$ é monótona não decrescente.
    \item \textbf{Falso}. Como $\lim_{n\to \infty} \frac{n}{n + 2024} = 1 \neq 0$, a série não pode ser convergente.
    \item \textbf{Falso}. A série $\sum_{n=1}^\infty 0$ converge para zero, mas a sequência $\{0, 0, 0, \ldots \}$ não é decrescente.
    \item \textbf{Verdadeiro}. $\frac{\diff}{\diff{x}}\left( \sum_{n=0}^\infty x^{2n}\right)
    = \sum_{n=0}^\infty \frac{\diff}{\diff{x}}\left( x^{2n} \right)
    = \sum_{n=0}^\infty 2n x^{2n}
    = 2\sum_{n=1}^\infty nx^{2n-1}$.
\end{enumerate}

\Exercise[title={2,0}] (PONTO EXTRA) Escreva a série de Maclaurin para a função $f(x) = \cos^2(x)$ e determine seu intervalo e raio de convergência.
\Answer Como $\cos(x) = \sum_{n=0}^\infty (-1)^{n} \frac{x^{2n}}{(2n)!}$, tem-se:

\begin{align*}
    \cos^2(x)
    & = \frac{1}{2} + \frac{1}{2}\cos(2 x)
      = \frac{1}{2} + \frac{1}{2} \left[\sum_{n=0}^\infty (-1)^{n} \frac{(2x)^{2n}}{(2n)!}\right]
      = \frac{1}{2} + \sum_{n=0}^\infty (-1)^{n} \frac{1}{2} \frac{2^{2n} x^{2n}}{(2n)!} \\
    & = \frac{1}{2} + \sum_{n=0}^\infty (-1)^{n} \frac{2^{2n-1} x^{2n}}{(2n)!}.
\end{align*}

Além disso, se $a_n = (-1)^{n} \frac{2^{2n-1} x^{2n}}{(2n)!}$, então
\[
\left|\frac{a_{n+1}}{a_{n}}\right|
= \frac{\frac{2^{2(n+1)-1} x^{2(n+1)}}{(2(n+1))!}}{\frac{2^{2n-1} x^{2n}}{(2n)!}}
= \frac{2^{2n+1} x^{2n+2}}{(2n+2)!} \frac{(2n)!}{2^{2n-1} x^{2n}}
= \frac{2^2 x^2}{(2n+2)(2n+1)}.
\]
Consequentemente, $\lim_{n\to \infty} \left|\frac{a_{n+1}}{a_{n}}\right| = 0 < 1$, para todo $x \in \R$. Pelo teste da razão, a série é convergente para todo $x \in \R$, isto é, seu intervalo de convergência é $(-\infty, \infty)$ e seu raio de convergência é infinito.

\end{ExerciseList}

\begin{center}
BOA PROVA E BOAS FÉRIAS!
\end{center}

\newpage
\restoregeometry
\section*{Respostas}
\shipoutAnswer
\end{document}
